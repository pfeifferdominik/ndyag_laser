\documentclass[twoside,colorback,accentcolor=tud4c,11pt]{tudreport}
\usepackage{ngerman}
\usepackage[utf8]{inputenc} 
\usepackage[T1]{fontenc}
\usepackage{siunitx}
\usepackage{hyperref}
\usepackage{units}
\usepackage{upgreek}
\usepackage{biblatex}
\usepackage{graphicx}
\usepackage{float}
\usepackage{subfigure}
\usepackage[figure]{hypcap}

\title{Zeeman-Effekt}
\subtitle{	\begin{tabular}{p{8cm}ll}
Benedikt Paul Schallmo   &   Dominik Pfeiffer \\ Matrikelnummer: 2686286  &   Matrikelnummer: 2913632       \\ email: \textaccent{ benediktschallmo@yahoo.de} & email: \textaccent{dominik@diepfeiffers.de}  
			\end{tabular} }
\subsubtitle{ \\Versuchsbetreuung :Dr. Mathias Sinther \\ Datum der Durchführung: 22.05.2027 \\ Abgabetermin: 12.06.2017    }
\institution{Institut für Angewandte Physik}
\sponsor{Hiermit erklären wir, dass wir die vorliegende Arbeit bzw. Leistung eigenständig, ohne fremde Hilfe und nur unter Verwendung der angegebenen Hilfsmittel angefertigt haben. Alle übernommenen Textstellen aus der Literatur beziehungsweise dem Internet sind als solche kenntlich gemacht. Diese Arbeit hat in gleicher oder ähnlicher Form noch keiner Prüfungsbehörde vorgelegen. \\\\ 
\begin{tabular}{lp{2em}lp{2em}l}
 \hspace{4cm}   && \hspace{4cm}  && \hspace{4cm}
 \\\cline{1-1}\cline{3-3}\cline{5-5}
 Ort, Datum     && Benedikt Schallmo && Dominik Pfeiffer
\end{tabular}  }


\dedication{}
\lowertitleback{}
\listfiles
    
\begin{document}

\maketitle 

\tableofcontents


\chapter{Einleitung und Ziel des Versuchs}
Ziel dieses Versuches ist es, die physikalischen Eigenschaften eines Halbleiterlasers und eines Festkörperlasers zu untersuchen. Hierfür soll zunächst das Verhalten des zum Pumpen verwendeten Halbleiterlasers näher betrachtet und dessen Kennlinie bezüglich Licht-/Pumpleistung aufgenommen werden. Anschließend wird der durch diesen Halbleiterlaser gepumpte Nd:YAG-Laser näher untersucht und die optimalen Arbeitskonfigurationen herausgearbeitet um zuletzt mittels eines KTP-Kristalls die Physik der Frequenzverdopplung zu nutzen und einige Kenngrößen dieser zu bestimmen.
\chapter{Physikalische Grundlagen}
\section{Grundlagen Laser}
\subsection{Laser allgemein}
\subsection{Halbleiter-Laser}
\subsection{Nd:YAG-Laser}
\section{Nichtlineare Optik}
\subsection{Grundlagen}
\subsection{Frequenzverdopplung}	
\chapter{Versuchsdurchführung und Beobachtung}
  
     	
\chapter{Auswertung}
\chapter{Fazit}
\chapter{Messdaten}


\chapter{Anhang}





		

\renewcommand{\bibname}{Literaturverzeichnis}
\begin{thebibliography}{Bak89}



\end{thebibliography} 	



\end{document} 